% ----------------------------------------------------------------------------------------------------------
% Kurfassung / Abstract
% ----------------------------------------------------------------------------------------------------------
\section*{Titel}
Entwicklung eines Lasertriangulationssensors zur Oberflächen-Rekonstruktion mit ROS2
\vfill
\subsubsection*{Kurzfassung}

Oberflächenrekonstruktion findet in der Industrie weitreichend ihre Anwendung. Sei es um kleine Bauteile zu untersuchen, größere Objekte auf Schäden hin zu prüfen oder auch ganze Landstriche zu vermessen. Dabei ist die Lasertriangulation eine gängige Methode, um 3D-Informationen der Oberfläche zu erhalten.
Die vorliegende Arbeit beschäftigt sich mit einem Open-Source entwickelten Lasertriangulationssensor. Dieser Sensor soll in einem an dem Institut für Technik und Informatik (ITI) durchgeführten Forschungsprojekt eingesetzt werden. Die Aufgabe dabei ist, dass der Sensor eine dreidimensionale Aufnahme von Holzscheiben erzeugt und dabei zusätzlich die Farbinformationen der Oberfläche liefert. In dieser Arbeit werden die Entwicklung und Umsetzung selbst, aber auch die mathematischen Grundlagen erläutert. Die Ergebnisse werden evaluiert und mit dem aktuellen Standard verglichen.

\section*{Title}
Development of a laser triangulation sensor for surface reconstruction with ROS2

\vfill
\subsubsection*{Abstract}

Surface reconstruction is widely used in industry. Be it for inspecting small components, checking larger objects for damage or even surveying entire areas. Laser triangulation is a common method to obtain 3D information of the surface. The present work deals with an open-source developed laser triangulation sensor. This sensor will be used in a research project at the Institut für Technik und Informatik (ITI). The task here is that the sensor generates a three-dimensional image of wooden slices and additionally provides the color information of the surface. In this paper the development and implementation itself, but also the mathematical basics are explained. The results are evaluated and compared with the current standard.
