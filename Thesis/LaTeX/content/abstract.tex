% ----------------------------------------------------------------------------------------------------------
% Kurfassung / Abstract
% ----------------------------------------------------------------------------------------------------------
\section*{Titel}
Entwicklung eines Lasertriangulationssensors zur Oberflächen-Rekonstruktion mit ROS2
\vfill
\subsubsection*{Kurzfassung}

Oberflächenrekonstruktion findet in der Industrie weitreichend ihre Anwendung. Sei es um kleine Bauteile zu untersuchen, größere Objekte auf Schäden zu prüfen oder auch ganze Landstriche zu vermessen. Dabei ist die Lasertriangulation eine gängige Methode, um 3D-Informationen zu erhalten. Diese Arbeit beschäftigt sich mit einem Open-Source entwickelten Lasertriangulationssensor. Dieser Sensor soll in einem an dem Institut für Technik und Informatik (ITI) durchgeführten Forschungsprojekt eingesetzt werden. Die Aufgabe dabei ist, von Holzscheiben eine 3D-Aufnahme mit zusätzlichen Farbinformationen zu erzeugen. In dieser Arbeit werden die Entwicklung und Umsetzung selbst, aber auch die mathematischen Grundlagen erläutert. Die Ergebnisse werden evaluiert und mit dem aktuellen Standard verglichen.

\section*{Title}
Development of a laser triangulation sensor for surface reconstruction with ROS2

\vfill
\subsubsection*{Abstract}

Surface reconstruction is widely used in industry. Be it for inspecting small components, checking larger objects for damage or even surveying entire areas. Laser triangulation is a common method for obtaining 3D information. This thesis deals with an open-source developed laser triangulation sensor. This sensor will be used in a research project at the Institut für Technik und Informatik (ITI). The task here is to generate a 3D image of wooden slices with additional color information. In this paper the development and implementation itself, but also the mathematical basics are explained. The results are evaluated and compared with the current standard.
