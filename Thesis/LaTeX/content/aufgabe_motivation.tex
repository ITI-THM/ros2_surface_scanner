\section{Aufgabe / Motivation}
Die grundsätzliche Aufgabe und Motivation entstand durch ein Projekt in Zusammenarbeit mit der Firma RINNTECH – „Technik zur Prüfung von Bäumen“. Die Firma bezeichnet sich selbst wie folgt: „Als Anwender und Entwickler verfügen wir über jahrelange Erfahrung und zahlreiche Patente auf dem Gebiet der Baum- und Holzanalyse. Für diese Anforderungen können wir Ihnen daher ausgereifte Technik und umfassenden Service anbieten“ [Vgl.: RIN]. Beispielanwendungen, für die Geräte und Software bereitgestellt werden sind zum Beispiel Bäume kontrollieren, Jahrringe analysieren, Holzkonstruktionen kontrollieren und Holzqualität und Zuwachs im Wald kontrollieren [Vgl.:  RIN]. Das zugrundeliegende Projekt wurde als Forschungs- und Entwicklungsprojekt zusammen mit dem Institut für Technik und Informatik (ITI) an der THM gestartet. Es trägt den Titel

\textit{"Entwicklung einer Messmethodik zur Ermöglichung einer schnellen Bestimmung von Holzart und -herkunft anhand von Jahrring- und Farbanalyse. Entwicklung der Messwerterfassung und -auswertung der neuen Messmethodik"}

und beschäftigt sich konkret mit der Jahrringanalyse. Durch diese soll Holzart und Herkunft bestimmt werden. RINNTECH verkauft das Produkt „LINTAP“ für diesen Zweck [Vgl.: LINTAP]. Das schon existierende Produkt soll in dem Projekt erweitert, optimiert und automatisiert werden. Die Grundidee besteht darin, dass ein Roboter-Arm mit einer hochauflösenden Kamera über das Objekt fährt und entsprechende Bilder aufnimmt, die für die Jahrringanalyse erforderlich sind. Hier wird eine Kamera eingesetzt, die von dem Roboterarm nah an das Holz herangeführt werden muss. Die Pfade, die der Roboter dementsprechend abfahren muss, müssen also mit hinreichender Genauigkeit errechnet werden. Dazu ist eine digitale Abbildung des Holzes unverzichtbar. Anhand einer rekonstruierten Oberfläche können die entsprechenden Pfade für den Roboter ohne Probleme errechnet werden. Es muss also initial eine Oberflächen-Rekonstruktion des Objektes stattfinden. An diesem Punkt setzt meine Arbeit an. Es wurde eine Software entwickelt, die mithilfe einer Kamera und einen Linienlaser einen 3D-Scan durchführt. Dabei werden die Tiefeninformationen und auch Farbinformationen aufgenommen und verarbeitet. Man erhält eine Punktewolke der gescannten Oberfläche die entsprechend gefärbt ist. Ein typischer Output für eine 3D-Kamera in der Industrie. Die Aufgabenstellung wurde noch etwas konkretisiert. Als Methode soll eine Lasertriangulation verwendet werden, dazu wurde die Kamer und der Linienlaser bereitgestellt. Zusätzlich soll nur Open-Source-Software verwendet werden und die Anwendung soll über ROS2 (Robot Operating System) laufen.