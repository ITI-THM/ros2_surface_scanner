% ----------------------------------------------------------------------------------------------------------
% Stand der Technik
% ----------------------------------------------------------------------------------------------------------
\section{Stand der Technik}\label{stand-der-technik}
	\subsection{Methodik}
	Lasertriangulation ist nicht die einzige Methode eine Oberflächen-Rekonstruktion durchzuführen. Die zentralen Technologien zu diesem Zweck sind Triangulation oder Time-of-Flight [Vgl.: SotA]. 
	Bei der Triangulation gibt es zwei Ansätze. Einen direkten über Structured Light. Dabei wird ein bekanntes Muster mit einem Laser oder Ähnlichem auf die Oberfläche projiziert. Eine Kamera nimmt das Muster als Bild auf, welches sich durch die variierende Höhe und Form der Oberfläche verzerrt. Diese Verzerrung wird als Anhaltspunkt verwendet, um die Unterschiedlichkeiten in der Höhe zu ermitteln. Lasertriangulation und der hier verwendete Lösungsansatz gehören zu dieser Möglichkeit. 
	Einen indirekten Ansatz der Triangulation bietet Stereo-Vision. Dabei werden zwei Kameras verwendet, die zwei aufgenommenen Bildern aus unterschiedlichen Positionen liefern. Ebenfalls wird ein fester Punkt benötigt, der auch mit einem Laser oder Ähnlichem im Bild projiziert werden kann. Über die zwei unterschiedlichen Bilder kann die Position ermittelt werden.
	
	Die Time-of-Flight-Technologie benutzt eine andere Lösungsmöglichkeit. Es wird Licht auf einen Punkt auf der Oberfläche gestrahlt. Dort wird es zurück reflektiert und von einem Sensor registriert. Dieser misst die verstrichene Zeit. Durch die bekannte Geschwindigkeit von Licht, kann über die gebrauchte Zeit der zurückgelegte Weg errechnet werden. Dieser entspricht der Höhe.

	\subsection{Geräte}
	Diese Methoden finden ihre Anwendung auch in der Industrie. Diverse Geräte und Anwendungen zur Oberflächen-Rekonstruktion sind bereits auf dem Markt. Die Rede ist von sogenannten 3D-Kameras bzw. RGB-D Kameras. Dabei steht RGB (Red, Green, Blue) für die Farbinformationen und das D (Depth) für die Tiefeninformation. Angefangen mit der von Microsoft entwickelten „Kinect“ über die „Intel RealSense“ zur „Google Tango“ folgen viele weitere Geräte. Diese sind nicht nur meist für einen geringen Preis verfügbar, sie könne auch die entsprechenden Pixelfarben in einer guten Auflösung aufnehmen. Zusätzlich geschieht die Aufnahme in Echtzeit, was bedeutet, dass sich die herausgegebene Punktewolke ändert, sobald sich die aufgenommene Oberfläche ändert bzw. die Kamera bewegt wird.
